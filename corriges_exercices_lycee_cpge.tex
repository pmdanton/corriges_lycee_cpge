\documentclass[a4paper,11pt]{article}
\usepackage[T1]{fontenc}
\usepackage[utf8]{inputenc}
\usepackage[french]{babel}
\usepackage{lmodern}
\usepackage{amsfonts}
\usepackage{amsmath}
\usepackage{hyperref}

\title{\textbf{Corrigés des exercices du lycée à la prépa scientifique}}
\author{}

\begin{document}

\newcommand{\exercice}[1]{\paragraph{Exercice #1.} \input{exercices/exercice_#1.tex}}

\maketitle
\begin{abstract}
Ce document fournit un corrigé aux exercices proposés par les professeurs des lycées Louis-Le-Grand et Henri-IV aux lycéennes et lycéens qui souhaitent se préparer aux CPGE scientifiques. Il n'est aucunement associé à ces établissements, et constitue un projet indépendant pour accompagner les élèves intéressés - à ce titre, il est fourni libre de droits, mais sans aucune garantie. Les énoncés des exercices sont disponibles \href{https://lycee-henri4.com/wp-content/uploads/2022/07/Mathematiques-terminale-Cpge.pdf}{ici}.\\

Un exercice n'a de valeur pédagogique que s'il est activement cherché: ce document peut alors servir de vérification, ou aider les élèves bloqués "pour de bon". \\

Pas moins de 561 exercices sont proposés: c'est considérable, et toutes les contributions pour les corriger sont donc bienvenues et appréciées! Pour participer ou consulter la dernière version, \href{https://github.com/pmdanton/corriges_lycee_cpge}{c'est ici}!

\end{abstract}

\newpage
\tableofcontents

\newpage
\section{Rédaction, modes de raisonnement}
\exercice{1}
\exercice{2}
\exercice{3}
\exercice{4}
\exercice{5}
\exercice{6}
\exercice{7}
\exercice{8}
\exercice{9}
\exercice{10}
\exercice{11}
\exercice{12}
\exercice{13}
\exercice{14}
\exercice{15}
\exercice{16}
\exercice{17}
\exercice{18}
\exercice{19}
\exercice{20}
\exercice{21}
\exercice{22}
\exercice{23}
\exercice{24}


\newpage
\section{Calcul algébrique}
\exercice{25}
\exercice{26}
\exercice{27}
\exercice{28}
\exercice{29}
\exercice{30}
\exercice{31}
\exercice{32}
\exercice{33}
\exercice{34}
\exercice{35}
\exercice{36}
\exercice{37}
\exercice{38}
\exercice{39}
\exercice{40}
\exercice{41}
\exercice{42}
\exercice{43}
\exercice{44}
\exercice{45}
\exercice{46}
\exercice{47}
\exercice{48}
\exercice{49}
\exercice{50}
\exercice{51}
\exercice{52}
\exercice{53}
\exercice{54}
\exercice{55}

\newpage
\section{Inégalités, inéquations, trinôme du second degré réel}
\exercice{56}
\exercice{57}
\exercice{58}
\exercice{59}
\exercice{60}
\exercice{61}
\exercice{62}
\exercice{63}
\exercice{64}
\exercice{65}
\exercice{66}
\exercice{67}
\exercice{68}
\exercice{69}
\exercice{70}
\exercice{71}
\exercice{72}
\exercice{73}
\exercice{74}
\exercice{75}
\exercice{76}
\exercice{77}
\exercice{78}
\exercice{79}
\exercice{80}
\exercice{81}
\exercice{82}
\exercice{83}
\exercice{84}
\exercice{85}
\exercice{86}
\exercice{87}
\exercice{88}
\exercice{89}
\exercice{90}
\exercice{91}
\exercice{92}
\exercice{93}
\exercice{94}

\newpage
\section{Trigonométrie}
\exercice{95}
\exercice{96}
\exercice{97}
\exercice{98}
\exercice{99}
\exercice{100}
\exercice{101}
\exercice{102}
\exercice{103}
\exercice{104}
\exercice{105}
\exercice{106}
\exercice{107}
\exercice{108}
\exercice{109}
\exercice{110}
\exercice{111}
\exercice{112}
\exercice{113}
\exercice{114}
\exercice{115}
\exercice{116}
\exercice{117}
\exercice{118}
\exercice{119}
\exercice{120}
\exercice{121}

\newpage
\section{Calcul des limites}
\exercice{122}
\exercice{123}
\exercice{124}
\exercice{125}
\exercice{126}
\exercice{127}
\exercice{128}
\exercice{129}
\exercice{130}
\exercice{131}
\exercice{132}
\exercice{133}
\exercice{134}
\exercice{135}
\exercice{136}
\exercice{137}
\exercice{138}
\exercice{139}
\exercice{140}
\exercice{141}
\exercice{142}
\exercice{143}
\exercice{144}

\newpage
\section{Dérivation}
\exercice{145}
\exercice{146}
\exercice{147}
\exercice{148}
\exercice{149}
\exercice{150}
\exercice{151}
\exercice{152}
\exercice{153}
\exercice{154}
\exercice{155}
\exercice{156}
\exercice{157}
\exercice{158}
\exercice{159}
\exercice{160}
\exercice{161}
\exercice{162}
\exercice{163}
\exercice{164}
\exercice{165}
\exercice{166}
\exercice{167}
\exercice{168}
\exercice{169}
\exercice{170}
\exercice{171}
\exercice{172}
\exercice{173}
\exercice{174}
\exercice{175}
\exercice{176}
\exercice{177}
\exercice{178}
\exercice{179}
\exercice{180}
\exercice{181}
\exercice{182}
\exercice{183}
\exercice{184}
\exercice{185}
\exercice{186}
\exercice{187}
\exercice{188}
\exercice{189}
\exercice{190}
\exercice{191}
\exercice{192}
\exercice{193}
\exercice{194}
\exercice{195}
\exercice{196}
\exercice{197}
\exercice{198}

\newpage
\section{Fonctions puissances}
\exercice{199}
\exercice{200}
\exercice{201}
\exercice{202}
\exercice{203}
\exercice{204}
\exercice{205}
\exercice{206}
\exercice{207}
\exercice{208}
\exercice{209}
\exercice{210}
\exercice{211}
\exercice{212}
\exercice{213}
\exercice{214}
\exercice{215}
\exercice{216}
\exercice{217}
\exercice{218}
\exercice{219}
\exercice{220}
\exercice{221}

\newpage
\section{Intégration}
\exercice{222}
\exercice{223}
\exercice{224}
\exercice{225}
\exercice{226}
\exercice{227}
\exercice{228}
\exercice{229}
\exercice{230}
\exercice{231}
\exercice{232}
\exercice{233}
\exercice{234}
\exercice{235}
\exercice{236}
\exercice{237}
\exercice{238}
\exercice{239}
\exercice{240}
\exercice{241}
\exercice{242}
\exercice{243}
\exercice{244}
\exercice{245}
\exercice{246}
\exercice{247}
\exercice{248}
\exercice{249}
\exercice{250}
\exercice{251}
\exercice{252}
\exercice{253}
\exercice{254}
\exercice{255}
\exercice{256}
\exercice{257}
\exercice{258}
\exercice{259}
\exercice{260}
\exercice{261}
\exercice{262}
\exercice{263}

\newpage
\section{Probabilités}
\exercice{264}
\exercice{265}
\exercice{266}
\exercice{267}
\exercice{268}
\exercice{269}
\exercice{270}
\exercice{271}
\exercice{272}
\exercice{273}
\exercice{274}
\exercice{275}
\exercice{276}
\exercice{277}
\exercice{278}
\exercice{279}
\exercice{280}
\exercice{281}
\exercice{282}
\exercice{283}
\exercice{284}
\exercice{285}
\exercice{286}
\exercice{287}
\exercice{288}
\exercice{289}
\exercice{290}
\exercice{291}
\exercice{292}
\exercice{293}
\exercice{294}
\exercice{295}
\exercice{296}
\exercice{297}
\exercice{298}
\exercice{299}
\exercice{300}
\exercice{301}
\exercice{302}
\exercice{303}
\exercice{304}
\exercice{305}
\exercice{306}
\exercice{307}
\exercice{308}
\exercice{309}
\exercice{310}
\exercice{311}
\exercice{312}
\exercice{313}
\exercice{314}
\exercice{315}
\exercice{316}
\exercice{317}
\exercice{318}
\exercice{319}
\exercice{320}
\exercice{321}
\exercice{322}
\exercice{323}
\exercice{324}
\exercice{325}
\exercice{326}
\exercice{327}
\exercice{328}
\exercice{329}

\newpage
\section{Nombres complexes}
\exercice{330}
\exercice{331}
\exercice{332}
\exercice{333}
\exercice{334}
\exercice{335}
\exercice{336}
\exercice{337}
\exercice{338}
\exercice{339}
\exercice{340}
\exercice{341}
\exercice{342}
\exercice{343}
\exercice{344}
\exercice{345}
\exercice{346}
\exercice{347}
\exercice{348}
\exercice{349}
\exercice{350}
\exercice{351}
\exercice{352}
\exercice{353}
\exercice{354}
\exercice{355}
\exercice{356}
\exercice{357}
\exercice{358}
\exercice{359}
\exercice{360}
\exercice{361}
\exercice{362}
\exercice{363}
\exercice{364}
\exercice{365}
\exercice{366}
\exercice{367}
\exercice{368}
\exercice{369}
\exercice{370}
\exercice{371}
\exercice{372}
\exercice{373}
\exercice{374}
\exercice{375}
\exercice{376}
\exercice{377}
\exercice{378}
\exercice{379}
\exercice{380}
\exercice{381}
\exercice{382}
\exercice{383}
\exercice{384}
\exercice{385}
\exercice{386}
\exercice{387}
\exercice{388}
\exercice{389}
\exercice{390}
\exercice{391}
\exercice{392}
\exercice{393}
\exercice{394}
\exercice{395}
\exercice{396}
\exercice{397}
\exercice{398}
\exercice{399}
\exercice{400}
\exercice{401}
\exercice{402}
\exercice{403}
\exercice{404}
\exercice{405}
\exercice{406}
\exercice{407}
\exercice{408}
\exercice{409}

\newpage
\section{Polynômes et équations algébriques}
\exercice{410}
\exercice{411}
\exercice{412}
\exercice{413}
\exercice{414}
\exercice{415}
\exercice{416}
\exercice{417}
\exercice{418}
\exercice{419}
\exercice{420}
\exercice{421}
\exercice{422}
\exercice{423}
\exercice{424}
\exercice{425}
\exercice{426}
\exercice{427}
\exercice{428}
\exercice{429}
\exercice{430}
\exercice{431}
\exercice{432}
\exercice{433}
\exercice{434}
\exercice{435}
\exercice{436}
\exercice{437}
\exercice{438}
\exercice{439}
\exercice{440}
\exercice{441}
\exercice{442}
\exercice{443}
\exercice{444}
\exercice{445}
\exercice{446}
\exercice{447}
\exercice{448}
\exercice{449}
\exercice{450}
\exercice{451}
\exercice{452}
\exercice{453}
\exercice{454}
\exercice{455}
\exercice{456}
\exercice{457}
\exercice{458}

\newpage
\section{Arithmétique}
\exercice{459}
\exercice{460}
\exercice{461}
\exercice{462}
\exercice{463}
\exercice{464}
\exercice{465}
\exercice{466}
\exercice{467}
\exercice{468}
\exercice{469}
\exercice{470}
\exercice{471}
\exercice{472}
\exercice{473}
\exercice{474}
\exercice{475}
\exercice{476}
\exercice{477}
\exercice{478}
\exercice{479}
\exercice{480}
\exercice{481}
\exercice{482}
\exercice{483}
\exercice{484}
\exercice{485}
\exercice{486}
\exercice{487}
\exercice{488}
\exercice{489}
\exercice{490}
\exercice{491}
\exercice{492}
\exercice{493}
\exercice{494}
\exercice{495}
\exercice{496}
\exercice{497}
\exercice{498}
\exercice{499}
\exercice{500}
\exercice{501}
\exercice{502}
\exercice{503}
\exercice{504}
\exercice{505}
\exercice{506}
\exercice{507}
\exercice{508}
\exercice{509}
\exercice{510}
\exercice{511}
\exercice{512}
\exercice{513}
\exercice{514}
\exercice{515}
\exercice{516}
\exercice{517}
\exercice{518}
\exercice{519}
\exercice{520}
\exercice{521}
\exercice{522}
\exercice{523}
\exercice{524}
\exercice{525}
\exercice{526}
\exercice{527}
\exercice{528}
\exercice{529}
\exercice{530}
\exercice{531}
\exercice{532}
\exercice{533}
\exercice{534}
\exercice{535}
\exercice{536}
\exercice{537}
\exercice{538}
\exercice{539}
\exercice{540}
\exercice{541}
\exercice{542}
\exercice{543}
\exercice{544}
\exercice{545}
\exercice{546}
\exercice{547}
\exercice{548}
\exercice{549}
\exercice{550}
\exercice{551}
\exercice{552}
\exercice{553}
\exercice{554}
\exercice{555}
\exercice{556}
\exercice{557}
\exercice{558}
\exercice{559}
\exercice{560}
\exercice{561}


\end{document}
