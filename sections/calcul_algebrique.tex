\paragraph{Exercice 36.} Soit $n\in\mathbb{N^*}$. Par linéarité de la somme:
\begin{eqnarray}
  \sum_{k=1}^n{(2k-1)} &=& 2\sum_{k=1}^n{k}-n \nonumber \\ 
                      &=& 2\frac{n(n+1)}{2}-n \nonumber \\ 
                      &=& n^2 \nonumber
\end{eqnarray}

\paragraph{Exercice 37.} On a $u_0=0$ (par convention, une somme vide vaut 0). Pour tout $n\in\mathbb{N}^*$:
\begin{eqnarray}
  u_n &=& \sum_{k=0}^n{u_k} - \sum_{k=0}^{n-1}{u_k} \nonumber \\ 
                      &=& \frac{n^2 + n}{3}- \frac{(n-1)^2 + (n-1)}{3}\nonumber \\ 
                      &=& \frac{2n}{3} \nonumber
\end{eqnarray}
Finalement, pour tout $n\in\mathbb{N}, u_n=\frac{2n}{3}$.

\paragraph{Exercice 38.} Chaque élément de la table s’écrit $i\times j$ avec $1\leq i,j\leq n$. La moyenne vaut donc:
\begin{eqnarray}
\frac{1}{n^2}\sum_{i=1}^n{\left(\sum_{j=1}^n{ij}\right)} &=& \frac{1}{n^2}\left(\sum_{i=1}^n{i}\right)\left(\sum_{j=1}^n{j}\right) \nonumber \\
&=& \frac{1}{n^2}\left(\frac{n(n+1)}{2}\right)^2 \nonumber \\
&=& \frac{(n+1)^2}{4} \nonumber 
\end{eqnarray}


\textit{Remarque}: la difficulté de l'exercice est de bien maîtriser le symbole $\Sigma$. Ici, on a pu mettre en facteur $i$ dans la somme sur $j$, puis mettre en facteur toute la somme sur $j$ dans la somme sur $i$.
\paragraph{Exercice 40.} a) Soient $n\in\mathbb{N}^*$ et $x\in\mathbb{R}$. Le cas $x=1$ est connu:
\[
\sum_{k=0}^n{k} = \frac{n(n+1)}{2}
\]

Supposons maintenant $x\neq 1$, ce qui va permettre de diviser par $1-x$ dans la suite. L’idée est de faire "apparaître" une dérivée: au lieu de $kx^k$, on voudrait $kx^{k-1}$, qui est la dérivée de $x\mapsto x^{k}$; on factorise donc par $x$:
\begin{eqnarray}
\sum_{k=0}^n{kx^k} &=& x\sum_{k=0}^n{kx^{k-1}} \nonumber \\
 &=& x\sum_{k=0}^n{\left( x^k \right)'} \nonumber \\
  &=& x \left( \sum_{k=0}^n{ x^k }\right)' \nonumber \\
  &=& x \left( \frac{x^{n+1}-1}{x-1} \right)' \nonumber \\
  &=& x \frac{(n+1)x^n(x-1)-(x^{n+1}-1)}{(x-1)^2} \nonumber \\
  &=& x \frac{nx^{n+1}-(n+1)x^n+1}{(x-1)^2} \nonumber
\end{eqnarray}

b) Pour tout $x\in ]-1,1[$, $nx^{n+1}\to 0$ et $(n+1)x^n\to 0$ quand $n\to+\infty$, donc:
\[
\forall x\in ]-1,1[, \quad \sum_{k=0}^n{kx^k} \mathop{\to}_{n\to +\infty} \frac{x}{(x-1)^2}
\]

\paragraph{Exercice 41.} Pour tout $n\in \mathbb{N}^*$:
\begin{eqnarray}
  u_{n+1}-u_n &=& \sum_{k=n+1}^{2n+2}{\frac{1}{k}} - \sum_{k=n}^{2n}{\frac{1}{k}} \nonumber \\
       &=& \frac{1}{2n+2} + \frac{1}{2n+1} - \frac{1}{n} \nonumber 
\end{eqnarray}
de plus, $ \frac{1}{2n+2} < \frac{1}{2n}$ et $ \frac{1}{2n+1} < \frac{1}{2n}$, donc $\frac{1}{2n+2} + \frac{1}{2n+2} < \frac{1}{n}$ ce qui prouve que $u_{n+1}-u_n<0$: la suite $(u_n)$ est strictement décroissante.
