\paragraph{Exercice 61.}  $\forall(x,y)\in\mathbb{R}^2$:
\begin{eqnarray}
  |x+y|\leq |x|+|y| &\Leftrightarrow& (x+y)^2\leq \left(|x|+|y|\right)^2 \nonumber \\
  &\Leftrightarrow& x^2+y^2 + 2xy\leq |x|^2+|y|^2 + 2|x||y| \nonumber \\
  &\Leftrightarrow& xy\leq |xy| \nonumber 
\end{eqnarray}
Ce qui est toujours vrai. De plus pour l'égalité, $xy=|xy| \Rightarrow xy\geq0$ et donc que $x$ et $y$ sont de même signe.

\paragraph{Exercice 62.} Pour tout $x\in\mathbb{R}$
\[
\sqrt{x^2} + \sqrt{(x-1)^2} = |x|+|1-x|
\]
On distingue trois cas:
\begin{itemize}
  \item $x<0 \Rightarrow |1-x| > 1 \Rightarrow |x|+|1-x| > 1$
  \item $x>1 \Rightarrow |x| > 1 \Rightarrow |x|+|1-x| > 1$
  \item $x\in[0,1]\Rightarrow |x|+|1-x|=x+(1-x)=1$
\end{itemize}
Finalement, l'ensemble des solutions est $[0,1]$.\\ \\

\textit{Interprétation géométrique et solution plus élégante}: la valeur absolue mesure la distance "usuelle" (euclidienne): $|x|=|x-0|$ est la distance de $x$ à 0, et $|x-1|$ est la distance de $x$ à 1. On cherche donc des points dont la distance à 0 plus la distance à 1 est identique. Si l'on prend le problème dans $\mathbb{R}^2$ au lieu de $\mathbb{R}$, c'est-à-dire dans le plan au lieu de sur une droite, c'est la définition d'une ellipse de foyers (0,0) et (1,0).
\begin{itemize}
  \item Ici, comme la constante est exactement la distance entre les foyers, l'ellipse est aplatie et devient le segment entre les foyers, c'est-a-dire $[0,1]$: on a donc une solution sans calcul, mais qui demande une certaine culture géométrique!
  \item Si la constante était strictement inférieure a la distance entre les foyers, il n'y aurait aucune solution, ni sur la droite, ni dans le plan. 
  \item Enfin, si la constante était strictement supérieure a la distance entre les foyers, les deux solutions réelles seraient l'intersection de l'ellipse avec l'axe des abscisses.% Traitons ce cas: pour un réel $\lambda>1$, déterminons l'ensemble des réels $x$ tels que $|x|+|1-x|=\lambda$. \\
 % On va astucieusement utiliser l’interprétation géométrique et l'axe de symétrie de l’ellipse d’équation $y=\frac{1}{2}$ (il y a un deuxième axe de symétrie, qu'on laisse a la lectrice!): si $x=\frac{1}{2}+z$ est solution pour $z>0$, alors $x'=\frac{1}{2}-z$ est aussi solution. On restreint donc le problème a $\mathbb{K}= \left[ \frac{1}{2},+\infty\right[$. Posons donc, pour tout $x\in\mathbb{K}$, $z=x-\frac{1}{2}>0$. Alors:
 % \begin{eqnarray}
 % |x|+|x-1|= \lambda &\Leftrightarrow& |z+\frac{1}{2}|+|z-\frac{1}{2}| = \lambda \nonumber \\
 %   &\Leftrightarrow& \left(z+\frac{1}{2}\right)^2+\left(z-\frac{1}{2}\right)^2 +2\left|z+\frac{1}%{2}\right|\left|z-\frac{1}{2}\right|= \lambda^2 \nonumber \\
 %   &\Leftrightarrow& \left(z+\frac{1}{2}\right)^2+\left(z-\frac{1}{2}\right)^2 +2\left|z^2-\frac{1}{4}\right|= \lambda^2 \quad (z>0)\nonumber 
%    \end{eqnarray}
\end{itemize}

\paragraph{Exercice 63.} Soit $n\in\mathbb{N}^*$. Pour tout entier non nul $k\leq 2n$, on a $\frac{1}{k}\geq \frac{1}{2n}$, donc:
\[
\sum_{k=n+1}^{2n}{\frac{1}{k}}\geq \sum_{k=n+1}^{2n}{\frac{1}{2n}}=\frac{1}{2}
\]
On utilise cette minoration pour montrer (avec un peu de travail pour recoller les morceaux!) que la somme des inverses des entiers tend vers l'infini.
\paragraph{Exercice 64.} Pour tout $n\in\mathbb{N}^*$ et $a\in]1,+\infty[$, on reconnaît la somme des termes d'une suite géométrique:
\[
\frac{a^n-1}{a-1} = \sum_{k=0}^{n-1}{a^k}
\]
Comme $a>1$, pour tout entier $k\leq n-1$, on a $a^k\leq a^{n-1}$, donc:
\[
\sum_{k=0}^{n-1}{a^k} \leq \sum_{k=0}^{n-1}{a^{n-1}} = na^{n-1}
\] 
Et finalement
\[
\frac{a^n-1}{a-1}  \leq na^{n-1}
\] 

\paragraph{Exercice 65.} Soient deux réels positifs $a$ et $b$. Par symétrie de l’énoncé, on peut supposer sans perte de généralité que $a\geq b$.
\begin{eqnarray}
  \left| \sqrt{a}-\sqrt{b} \right| \leq \sqrt{|a-b|} &\Leftrightarrow &  \left( \sqrt{a}-\sqrt{b} \right)^2 \leq |a-b| \nonumber \\
   &\Leftrightarrow &  a+b-2\sqrt{ab} \leq |a-b| \nonumber 
\end{eqnarray}
Puisqu'on a supposé $a \geq b \geq 0$, $\sqrt{ab}\geq b$ et $|a-b|=a-b$:
\begin{eqnarray}
  \left| \sqrt{a}-\sqrt{b} \right| \leq \sqrt{|a-b|} &\Leftrightarrow &  a+b-2b \leq a-b \nonumber \\
  &\Leftrightarrow &  a-b \leq a-b \nonumber
\end{eqnarray}
Ce qui prouve le résultat.\\ \\
\textit{Alternative}: on peut aussi utiliser le fait que pour $a$ et $b$ positifs, $|a-b|=\max(a,b)-\min(a,b)$, et $\min(a,b) \leq \sqrt{ab} \leq \max(a,b)$.
 
\paragraph{Exercice 66.} On sait que pour tout $(a,b)\in\mathbb{R}^2$, $(a+b)(a-b)=a^2-b^2$. En prenant $a=\sqrt{k+1}$ et $b=\sqrt{k}$ dans chaque terme de la somme, il vient:
\begin{eqnarray}
  \sum_{k=1}^n{\frac{1}{\sqrt{k}+\sqrt{k+1}}} \geq 2022 &\Leftrightarrow & \sum_{k=1}^n{\frac{\sqrt{k+1}-\sqrt{k}}{(k+1)-k}} \geq 2022 \nonumber \\
  &\Leftrightarrow & \sum_{k=1}^n{\left(\sqrt{k+1}-\sqrt{k}\right)} \geq 2022 \nonumber \\
  &\Leftrightarrow &  \sqrt{n+1} - 1 \geq 2022 \quad \textrm{par téléscopage}\nonumber \\
  &\Leftrightarrow &  n\geq 2023^2-1 \nonumber 
\end{eqnarray}
 Finalement, le plus petit entier solution est $2023^2-1$.

\paragraph{Exercice 68.} a) pour tout $n\in\mathbb{N}^*$ et $0\leq m \leq n-1$, on rappelle que:
\[
\binom{n}{m} = \frac{n!}{m!(n-m)!}
\]
et que la factorielle satisfait $(m+1)!=m!(m+1)$, donc en simplifiant:
\[
\frac{\binom{n}{m+1} }{\binom{n}{m} } = \frac{n-m}{m+1} = 1+\frac{n-2m-1}{m+1}
\]
ce qui montre que 
\[
\frac{\binom{n}{m+1}}{\binom{n}{m}}  \geq 1 \Leftrightarrow n-2m-1 \geq 0 \Leftrightarrow  m \leq \left\lfloor \frac{n-1}{2} \right\rfloor
\]
b) D’après le point a), on a toute de suite:
\[
\binom{n}{0} \leq \binom{n}{1} \leq \cdots \leq \binom{n}{\left\lfloor \frac{n-1}{2} \right\rfloor+1}
\]
or $\left\lfloor \frac{n-1}{2} \right\rfloor+1 =  \left\lfloor \frac{n-1}{2} +1\right\rfloor = \left\lfloor \frac{n+1}{2} \right\rfloor \geq \left\lfloor \frac{n}{2} \right\rfloor$, ce qui permet de retrouver le résultat de l’énonce.\\ \\
c) On rappelle la formule du binôme: 
\[
\forall n \in \mathbb{N}^*, \forall (x,y)\in\mathbb{R}^2, \quad (x+y)^n=\sum_{k=0}^n{\binom{n}{k}x^ky^{n-k}} 
\]
En prenant $x=y=1$, et comme les coefficients binomiaux sont positifs\footnote{c'est important de le préciser, sinon la somme pour être inférieure à l'un de ses termes!}, on a tout de suite la majoration:
\[
\binom{n}{\left\lfloor \frac{n}{2} \right\rfloor} \leq \sum_{k=0}^n{\binom{n}{k}} = (1+1)^n = 2^n
\]
Reste la minoration: en se rappelant que $\binom{n}{k}=\binom{n}{n-k}$, on voit que $\binom{n}{\left\lfloor \frac{n}{2} \right\rfloor}$ majore en fait tous les $\binom{n}{k}$ pour $0\leq k \leq n$. Comme il y a exactement $n+1$ termes dans la formule du binôme, on a:
\[
2^n = \sum_{k=0}^n{\binom{n}{k}} \leq (n+1)\binom{n}{\left\lfloor \frac{n}{2} \right\rfloor}
\]
ce qui prouve finalement que $\frac{2^n}{n+1}\leq \binom{n}{\left\lfloor \frac{n}{2} \right\rfloor} \leq 2^n $.

\paragraph{Exercice 69.} a) Soit $x\in\mathbb{R}$ et $k\in\mathbb{N}^*$. On note $f_k:x\mapsto |x-k|$, fonction définie de $\mathbb{R}$ dans $\mathbb{R}$. \\
Elle est dérivable sur $\mathbb{R}\backslash{\{k\}}$ et sa dérivée vaut -1 si $x<k$, et +1 si $x>k$.\\
 Par linéarité de la dérivation, on a donc que $f=\sum_{k=1}^n{f_k}$ est dérivable sur $\mathbb{K}=\mathbb{R}\backslash\{1,2,\cdots,2022\}$,
 et sa dérivée au point $x\in\mathbb{K}$ vaut:
 \[
 (\textrm{nombre d'entiers naturels inferieurs à }x) - (\textrm{nombre d'entiers superieurs à } x \textrm{ et inferieurs à 2022})
 \]
 De façon informelle, on voit que ce nombre est négatif jusqu’à avoir autant d'entiers avant et après $x$ ($x<1011$), qu'alors il s'annule ($1011 < x < 1012$), et redevient positif ($x>1012$). Plus formellement, on écrit pour tout $x\in\mathbb{K}$:
\begin{eqnarray}
  f'(x) &=& -2022 \quad \textrm{si } x < 1 \nonumber \\
  f'(x) &=& +2022 \quad \textrm{si } x > 2022 \nonumber \\
  f'(x) &=& \lfloor x \rfloor - (2022-\lfloor x \rfloor)=2\lfloor x \rfloor -2022 \quad \textrm{si } x \in \mathbb{K}\cap ]1,2022[ \nonumber 
\end{eqnarray}
 Étudions le signe de $f'$ sur $\mathbb{K}\cap ]1,2022[$ pour étudier les variations de $f$. On rappelle que $x\in\mathbb{K}$, donc $x$ n'est jamais entier:
 \begin{eqnarray}
2\lfloor x \rfloor -2022 < 0 &\Leftrightarrow & \lfloor x \rfloor < 1011 \Leftrightarrow x < 1011 \nonumber \\
2\lfloor x \rfloor -2022 = 0 &\Leftrightarrow & \lfloor x \rfloor = 1011 \Leftrightarrow x \in ]1011, 1012[ \nonumber \\
2\lfloor x \rfloor -2022 > 0 &\Leftrightarrow & \lfloor x \rfloor > 1011 \Leftrightarrow x > 1012 \nonumber
 \end{eqnarray}

En recollant avec les autres intervalles, on conclut que $f$ est strictement décroissante sur $]-\infty, 1011[$, constante sur $]1011, 1012[$, et strictement croissante sur $[1012, +\infty[$. \\ \\
b) D’après ce qui précède, et comme $f$ est continue sur $\mathbb{R}$, $f$ atteint son minimum en tout point de l'intervalle $]1011,1012[$. Considérons donc $m\in]1011,1012[$, et regroupons astucieusement les termes: à tout indice $1 \leq k\leq 1011$ on associe l'indice $k'=2023-k$. On observe que $k<m<k'$ donc $|m-k|+|m-k'|=|k'-k|=|2023-2k|$, donc:
\[
f(m)=\sum_{k=1}^{1011}{(2023-2k)}
\]
Effectuons le changement d'indice $j=1012-k$ (donc $k=1012-j$, ce qui revient simplement à sommer les termes "à l'envers"):
\[
f(m)=\sum_{k=j}^{1011}{(2j-1)}=1011^2
\]
car on se souvient, bien entendu, de la formule de la somme des nombres impairs! Pour rappel, elle est démontrée dans l'exercice 36.\\
Finalement, le minimum de la fonction vaut $1011^2$.

\paragraph{Exercice 70.} Pour tout $x\in\mathbb{R}$, on note $F(x)=x-\lfloor x \rfloor \in [0,1[$, appelée \textit{partie fractionnaire} de $x$. Alors $2x=2\lfloor x \rfloor + 2F(x)$. Deux cas sont possibles:
\begin{itemize}
  \item $F(x)<\frac{1}{2}\Rightarrow 2F(x)\in [0,1[ \Rightarrow \lfloor 2x \rfloor = 2\lfloor x \rfloor $
  \item $F(x)\in \left[\frac{1}{2}, 1\right[ \Rightarrow 2F(x)\in [1,2[ \Rightarrow \lfloor 2x \rfloor = 2\lfloor x \rfloor + 1$
\end{itemize}
Finalement, $\lfloor 2x \rfloor - 2\lfloor x \rfloor\in \{0,1\}$. 


\paragraph{Exercice 72.} On va utiliser les notations et le résultat de l'exercice 70. On introduit en plus la notation "indicatrice" pour tout énoncé $p$: $1_{p}$ qui vaut 1 si $p$ est vrai et 0 sinon. 
Dans l'exercice 70, on a vu que pour tout $x \in \mathbb{R}$,
 $\lfloor 2x \rfloor = 2\lfloor x \rfloor + 1_{F(x)\geq \frac{1}{2}}$
Comme $\lfloor x+y \rfloor = \lfloor x \rfloor  + \lfloor y \rfloor + \lfloor F(x)+F(y) \rfloor$, on obtient après simplification:
\[
\lfloor 2x \rfloor + \lfloor 2y \rfloor  - \lfloor x+y \rfloor  - \lfloor x \rfloor  - \lfloor x \rfloor = 1_{F(x)\geq \frac{1}{2}} + 1_{F(y)\geq \frac{1}{2}} - \lfloor F(x)+F(y) \rfloor 
\]
On remarque ensuite que $\lfloor F(x)+F(y) \rfloor = 1_{\min(F(x),F(y))\geq \frac{1}{2}}$ pour conclure: le résultat vaut $1+1-1=1$ si $\min(F(x), F(y)\geq \frac{1}{2}$, et $1_{F(x)\geq \frac{1}{2}} + 1_{F(y)\geq \frac{1}{2}} = 1_{\max(F(x), F(y))\geq \frac{1}{2}} \in \{0,1\}$ sinon. Dans tous les cas, $\lfloor 2x \rfloor + \lfloor 2y \rfloor  - \lfloor x+y \rfloor  - \lfloor x \rfloor  - \lfloor x \rfloor \in \{0,1\}$.

\paragraph{Exercice 83.} Pour $m\in\mathbb{R}$, le discriminant vaut:
\[
\Delta(m)=m^2-4=(m-2)(m+2)
\]
Donc:
\begin{itemize}
  \item Si $m\in\{-2,+2\}$, $\Delta(m)=0$ et $p_m$ admet une racine réelle double.
  \item Si $|m|<2$, $\Delta(m)<0$ et $p_m$ n'admet aucune racine réelle.
  \item Si $|m|>2$, $\Delta(m)>0$ et $p_m$ admet deux racines réelles distinctes.
\end{itemize}
