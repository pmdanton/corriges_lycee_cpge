\paragraph{Exercice 145.} Appliquer le cours.

\paragraph{Exercice 146.} D’après le cours, $h'=f' \cdot (g' \circ f)$, donc en dérivant le produit de deux fonctions:
\[
h''=f''\cdot (g' \circ f) + (f')^2 \cdot (g'' \circ f) 
\]

\paragraph{Exercice 147.} Prouvons l’énonce par récurrence sur $n\in\mathbb{N}$.
\begin{itemize} 
\item \textit{Initialisation}: La propriété est triviale au rang $n=0$ ($\cos=\cos$).\\
\item \textit{Hérédité}: Supposons la propriété vérifiée au rang $n\in\mathbb{N}$, alors pour tout $ x \in \mathbb{R}$:
\begin{eqnarray}
 \cos^{(n+1)}(x) &=& \left(\cos^{(n)}(x)\right)' \nonumber \\
  &=&  \left(\cos \left( x+\frac{n\pi}{2} \right) \right)' \nonumber \\
  &=&  -\sin \left( x+\frac{n\pi}{2} \right) \nonumber \\
  &=&  \cos \left( x+\frac{n\pi}{2} +\frac{\pi}{2}\right) \nonumber \\
  &=&  \cos \left( x+\frac{(n+1)\pi}{2}\right) \nonumber 
\end{eqnarray}
donc la propriété est vraie au rang $n+1$.
\end{itemize}
\textit{Remarque}: on a utilisé le fait que $\cos'=-\sin$, et que: $\forall \theta\in\mathbb{R}, \, \sin(\theta)=-\cos \left(\theta+\frac{\pi}{2}\right)$, qui se retrouve facilement en dessinant le cercle trigonométrique.

\paragraph{Exercice 148.} Soit $n \in \mathbb{N}^*$. Montrons par récurrence finie que pour tout entier $0 \leq k \leq n$, $\forall x\in \mathbb{R}, \, g^{(k)}(x)=a^kf^{(k)}(ax+b)$.
\begin{itemize}
  \item \textit{Initialisation}: pour $k=0$, $g(x)=f(ax+b)$ pour tout réel $x$ par définition.
  \item \textit{Hérédité}: supposons la propriété vraie au rang $k \leq n-1$, alors pour tout $x \in \mathbb{R}$:
  \begin{eqnarray}
    g^{(k+1)}(x) &=& \left( g^{(k)}(x) \right)'  \nonumber \\
                &=& \left( a^kf^{(k)}(ax+b) \right)'  \nonumber \\
                &=& a^k \left( f^{(k)}(ax+b) \right)'  \nonumber \\
                &=& a^{k+1} f^{(k+1)}(ax+b)  \nonumber 
  \end{eqnarray}
  Ce qui prouve le résultat.
\end{itemize}

\textit{Remarque:} il est utile de calculer les premières dérivées à la main pour avoir l'intuition du résultat.

\paragraph{Exercice 149.} Montrons par récurrence finie que pour tout entier $n\in \mathbb{N}$, et pour $x\in \mathbb{R}^*, \, f^{(n)}(x)=(-1)^nn!x^{-(n+1)}$.
\begin{itemize}
  \item \textit{Initialisation}: pour $n=0$, $f(x)=x^{-1}$ pour tout réel non nul $x$ par définition.
  \item \textit{Hérédité}: supposons la propriété vraie au rang $n\in \mathbb{N}1$, alors pour tout $x \in \mathbb{R}^*$:
  \begin{eqnarray}
    f^{(n+1)}(x) &=& \left( f^{(n)}(x) \right)'  \nonumber \\
                &=& \left( (-1)^nn!x^{-(n+1)} \right)'  \nonumber \\
                &=& (-1)^nn! \left( x^{-(n+1)} \right)'  \nonumber \\
                &=& (-1)^nn!(-1)(n+1) x^{-(n+1)-1} \nonumber \\
                &=& (-1)^{n+1}(n+1)!x^{-((n+1)+1)} \nonumber 
  \end{eqnarray}
  Ce qui prouve le résultat.
\end{itemize}

\textit{Remarque:} il est utile de calculer les premières dérivées à la main pour avoir l'intuition du résultat.

\paragraph{Exercice 150.} Montrons l'existence de $P_n$ par récurrence sur $n\in\mathbb{N}$.
\begin{itemize}
  \item \textit{Initialisation}: pour $n=0$, on a clairement $P_0=1$.
  \item \textit{Hérédité}: supposons la propriété vraie au rang $n\in \mathbb{N}1$, alors pour tout $x \in \mathbb{R}^*$:
  \begin{eqnarray}
    f^{(n+1)}(x) &=& \left( f^{(n)}(x) \right)'  \nonumber \\
                &=& \left( P_n(x)e^{-x^2} \right)'  \nonumber \\
                &=& P_n'(x)e^{-x^2}-2xP_n(x)e^{-x^2}  \nonumber \\
                &=& \left( P_n'(x) - 2xP_n(x) \right)e^{-x^2} \nonumber 
  \end{eqnarray}
  or $P_n'$ et $x\mapsto-2xP_n(x)$ sont des polynômes, donc leur somme l'est aussi.
\end{itemize}
Calculons $P_0,P_1,P_2$: on a vu que $P_0=1$. Puis, en utilisant la relation de récurrence, pour tout $x\in\mathbb{R}$, $P_1(x)=-2x$, et $P_2(x)=(-2)-2x(-2x)=4x^2-2=2(x^2-1)$.

\textit{Pour aller plus loin:} montrer que $P_n$ est de degré $n$ pour tout $n\in\mathbb{N}$.
