Montrons l'existence de $P_n$ par récurrence sur $n\in\mathbb{N}$.
\begin{itemize}
  \item \textit{Initialisation}: pour $n=0$, on a clairement $P_0=1$.
  \item \textit{Hérédité}: supposons la propriété vraie au rang $n\in \mathbb{N}$, alors pour tout $x \in \mathbb{R}^*$:
  \begin{eqnarray}
    f^{(n+1)}(x) &=& \left( f^{(n)}(x) \right)'  \nonumber \\
                &=& \left( P_n(x)e^{-x^2} \right)'  \nonumber \\
                &=& P_n'(x)e^{-x^2}-2xP_n(x)e^{-x^2}  \nonumber \\
                &=& \left( P_n'(x) - 2xP_n(x) \right)e^{-x^2} \nonumber 
  \end{eqnarray}
  or $P_n'$ et $x\mapsto-2xP_n(x)$ sont des polynômes, donc leur somme l'est aussi.
\end{itemize}
Calculons $P_0,P_1,P_2$: on a vu que $P_0=1$. Puis, en utilisant la relation de récurrence, pour tout $x\in\mathbb{R}$, $P_1(x)=-2x$, et $P_2(x)=(-2)-2x(-2x)=4x^2-2$.\\

\textit{Pour aller plus loin:} calculer le degré de $P_n$ pour tout $n\in\mathbb{N}$.
