Soient $a,b,c,d\in\mathbb{Q}$ tels que $a+b\sqrt{2}=c+d\sqrt{2}$, donc $a-c=(d-b)\sqrt{2}$. Supposons $d-b\neq 0$, alors $\frac{a-c}{d-b}=\sqrt{2}$, et $\frac{a-c}{d-b} \in \mathbb{Q}$: c'est une contradiction, puisque $\sqrt{2}\notin \mathbb{Q}$. On a donc $d=b$, ce qui entraîne immédiatement que $a=c$.
