a) Soit $u\in\mathbb{Q}$ et $v\notin \mathbb{Q}$. Supposons $u+v\in\mathbb{Q}$: alors $v=(u+v)-u$ est la différence de deux nombre rationnels, donc $v$ est rationnel: contradiction.\\
b) Soit $u\in\mathbb{Q}^*$ et $v\notin \mathbb{Q}$. Supposons $uv\in\mathbb{Q}$: comme $u\neq 0$, il est inversible et son inverse est rationnel. alors $v=uv\frac{1}{u}$ est le produit de deux rationnels, donc $v\in\mathbb{Q}$: contradiction.\\
c) $\pm \sqrt{2}\notin\mathbb{Q}$ mais $\sqrt{2}+(-\sqrt{2})=0\in \mathbb{Q}$. On a aussi $\sqrt{2}+\sqrt{2}=2\sqrt{2}\notin\mathbb{Q}$. Pour le produit, $\sqrt{2}\sqrt{2}=4\in \mathbb{Q}$, mais $\sqrt{2}\sqrt{3}\notin \mathbb{Q}$ car 6 n'est pas un carré parfait\footnote{cf exercice 18}.
