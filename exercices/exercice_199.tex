a) Pour tout $\alpha \in \mathbb{R}$ et tout $x \in \mathbb{R}^*_+$, 
\[
\varphi_{\alpha}'(x)=\alpha x^{\alpha-1}, \quad \varphi_{\alpha}''(x)=\alpha(\alpha-1)x^{\alpha-2}
\] 
$\varphi_{\alpha}$ est convexe si et seulement si $\varphi_{\alpha}''\geq 0$, ce qui équivaut \`a $\alpha(\alpha-1)\geq 0$ donc $\alpha\leq 0$ ou $\alpha \geq 1$.\\ \\
b) Soit $\alpha > 1$, alors $\varphi_{\alpha}$ est convexe. Une fonction convexe est au-dessus de ses tangentes, en particulier $\varphi_{\alpha}$ est au-dessus de sa tangente en $1$, qui a pour équation pour $t>0$:
\[
D_1:t\mapsto \varphi_{\alpha}(1)+\varphi_{\alpha}'(1)\cdot (t-1) = 1+\alpha(t-1)
\]
Ainsi: 
\[
\forall t\in ]0,+\infty[, \quad t^\alpha \geq 1 + \alpha(t-1)
\]
On retrouve bien l’énoncé en posant $x=t-1$, avec $x>-1$.
