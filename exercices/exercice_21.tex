$\sqrt{6}$ est irrationnel car 6 n'est pas un carré parfait (cf exercice 18). Supposons $\sqrt{2}+\sqrt{3}\in \mathbb{Q}$, alors il existe des entiers $a,b$ avec $b\neq 0$ tels que $\sqrt{2}+\sqrt{3}=\frac{a}{b}$, donc en élevant au carré: $2 + 2\sqrt{6} + 3 = \frac{a^2}{b^2}$, et $\sqrt{6}=\frac{a^2}{2b^2}-\frac{5}{2} \in \mathbb{Q}$: contradiction.
