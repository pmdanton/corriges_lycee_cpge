Pour tout $x\in\mathbb{R}$, on note $F(x)=x-\lfloor x \rfloor \in [0,1[$, appelée \textit{partie fractionnaire} de $x$. Alors $2x=2\lfloor x \rfloor + 2F(x)$. Deux cas sont possibles:
\begin{itemize}
  \item $F(x)<\frac{1}{2}\Rightarrow 2F(x)\in [0,1[ \Rightarrow \lfloor 2x \rfloor = 2\lfloor x \rfloor $
  \item $F(x)\in \left[\frac{1}{2}, 1\right[ \Rightarrow 2F(x)\in [1,2[ \Rightarrow \lfloor 2x \rfloor = 2\lfloor x \rfloor + 1$
\end{itemize}
Finalement, $\lfloor 2x \rfloor - 2\lfloor x \rfloor\in \{0,1\}$. 

