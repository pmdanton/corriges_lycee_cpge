On vérifie (par exemple par récurrence) que pour tout $n\in\mathbb{N}$, $t_n>0$, donc on peut considérer la suite définie pour $u_n=\ln(t_n)$ pour $n\in\mathbb{N}$. Alors:
\begin{eqnarray*}
  u_{n+1} &=& \ln(t_{n+1}) \\
  &=& \ln\left(\frac{\sqrt{t_{n}}}{e}\right) \\
  &=& \frac{1}{2}\ln(t_{n})-1 \\
  &=& \frac{1}{2}u_n-1
\end{eqnarray*}
Ainsi, $(u_n)$ est une suite arithmético-géométrique. D’après l’exercice 3, et avec les mêmes notations, on obtient $\ell = -2$ et pour tout $n$:
\[
u_n = \left(\frac{1}{2}\right)^{n-1}-2
\]
et par conséquent:
\[
t_n = e^{-2}e^{-\frac{1}{2^{n-1}}}
\]
et l'on conclut que $t_n\to e^{-2}$ quand $n$ tend vers $+\infty$.  
