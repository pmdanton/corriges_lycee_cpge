Il suffit d'adapter la preuve de l’irrationalité de $\sqrt{2}$: supposons que $\sqrt{3}\in\mathbb{Q}$, alors il existe des entiers naturels $a$ et $b$ premiers entre eux tels que $\sqrt{3}=\frac{a}{b}$, ce qui implique $a^2=3b^2$. Comme 3 est premier et divise $a^2$, d’après le lemme d'Euclide\footnote{\textit{Lemme d'Euclide}: soient $a$ et $b$ deux entiers. Si un nombre premier $p$ divise le produit $ab$, alors $p$ divise $a$ ou $p$ divise $b$. Le \textit{lemme de Gauss} en est une généralisation: pour trois entiers $a,b,c$, si $a$ divise le produit $bc$ et $a$ est premier avec $b$, alors $a$ divise $c$.} 3 divise $a$. Il existe donc un entier $k$ tel que $a=3k$, et $3^2k^2=3b^2$, soit encore $b^2=3k^2$: on voit que 3 divise également $b$, en contradiction avec l’hypothèse que $a$ et $b$ sont premiers entre eux.\\ \\
Cette preuve s’étend directement pour montrer que pour tout nombre premier $p$, $\sqrt{p}\notin \mathbb{Q}$. Plus généralement, on peut montrer que pour tout entier $n\in \mathbb{N}$, $\sqrt{n}\in\mathbb{Q}\Rightarrow \sqrt{n}\in\mathbb{N}$: si un entier n'est pas un carré parfait, alors sa racine carrée est irrationnelle. Supposons $\sqrt{n}\in\mathbb{Q}$, alors il existe des entiers $a$ et $b$ premiers entre eux, avec $b\neq 0$, tels que $a^2=nb^2$. Comme $a$ et $b$ sont premiers entre eux, $a^2$ et $b^2$ le sont aussi. Comme $a^2$ divise $nb^2$, d’après le lemme de Gauss $a^2$ divise $n$. D'autre part, $a^2 \geq n$, donc $a^2=n$ et $b=1$: $n$ est bien un carré parfait.
