Pour tout $x\in\mathbb{R}$:
\begin{eqnarray*}
  2\cos(2x)+4\cos(x)+3 &=& 2\left( 2\cos^2(x)-1\right)+4\cos(x)+3 \\
  &=& 4\cos^2(x)+4\cos(x)+1 \\
  &=& \left(\cos(x)+\frac{1}{2}\right)^2
\end{eqnarray*}
 on en déduit que $2\cos(2x)+4\cos(x)+3 \geq 0$, et pour l'égalité:
 \begin{eqnarray*}
   2\cos(2x)+4\cos(x)+3 = 0 &\Leftrightarrow & \left(\cos(x)+\frac{1}{2}\right)^2 = 0 \\
   & \Leftrightarrow & \cos(x) = -\frac{1}{2} \\
   & \Leftrightarrow & x\in \left\lbrace -\frac{2\pi}{3}, \frac{2\pi}{3} \right\rbrace \textrm{ modulo } 2\pi
 \end{eqnarray*}
