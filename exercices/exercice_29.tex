Soit $(x,y)\in\mathbb{K}^2$. Il existe donc $(a,b,c,d)\in\mathbb{Q}^4$ tel que $x=a+b\sqrt{2}$ et $y=c+d\sqrt{2}$. Alors:
\begin{itemize}
  \item $x+y=(a+c)+(b+d)\sqrt{2}$ et $(a+c, b+d)\in\mathbb{Q}^2$, donc $x+y\in\mathbb{K}$
  \item $x-y=(a-c)+(b-d)\sqrt{2}$ et $(a-c, b-d)\in\mathbb{Q}^2$, donc $x+y\in\mathbb{K}$. On peut aussi simplement remarquer que $y\in\mathbb{K}\Rightarrow -y\in\mathbb{K}$ et se ramener au cas précédent.
  \item $xy=(ac+2bd)+(ad+bc)\sqrt{2}$ et $(ac+2bd, ad+bc)\in\mathbb{Q}^2$, donc $xy\in\mathbb{K}$
  \item    Supposons $x\neq 0$, alors on a aussi $a-b\sqrt{2}\neq 0$: 
    \begin{itemize}
      \item si $b=0$, alors $a\neq 0$ et le résultat est acquis.
      \item si $b\neq 0$, on aurait $\sqrt{2}=a/b\in\mathbb{Q}$, contradiction.
    \end{itemize}
  Reprenons l’énoncé, et utilisons la quantité conjuguée $a-b\sqrt{2}\neq 0$:
  \[
  \frac{1}{a+b\sqrt{2}}=\frac{a-b\sqrt{2}}{a^2-2b^2} = \left(\frac{a}{a^2-2b^2} \right) + \left(\frac{-b}{a^2-2b^2} \right)\sqrt{2}
  \]
  et $\frac{a}{a^2-2b^2}\in\mathbb{Q}$, $\frac{-b}{a^2-2b^2} \in\mathbb{Q}$, donc $\frac{1}{x}\in \mathbb{K}$. \\
\end{itemize}
\textit{Remarque}: l'ensemble $\mathbb{K}$ est souvent noté $\mathbb{Q}[\sqrt{2}]$, et l'on vient de démontrer qu'il est stable par addition et multiplication, et que chaque élément possède un opposé et un inverse (sauf 0). On verra en première année de prépa que ces propriétés sont liées à la structure d'anneau, et que cet exercice montre en fait que $\mathbb{Q}[\sqrt{2}]$ est un sous-anneau de $(\mathbb{R}, +, \times)$.
