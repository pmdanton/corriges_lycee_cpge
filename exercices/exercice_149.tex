Montrons par récurrence que pour tout entier $n\in \mathbb{N}$, et pour $x\in \mathbb{R}^*, \, f^{(n)}(x)=(-1)^nn!x^{-(n+1)}$.
\begin{itemize}
  \item \textit{Initialisation}: pour $n=0$, $f(x)=x^{-1}$ pour tout réel non nul $x$ par définition.
  \item \textit{Hérédité}: supposons la propriété vraie au rang $n\in \mathbb{N}$, alors pour tout $x \in \mathbb{R}^*$:
  \begin{eqnarray}
    f^{(n+1)}(x) &=& \left( f^{(n)}(x) \right)'  \nonumber \\
                &=& \left( (-1)^nn!x^{-(n+1)} \right)'  \nonumber \\
                &=& (-1)^nn! \left( x^{-(n+1)} \right)'  \nonumber \\
                &=& (-1)^nn!(-1)(n+1) x^{-(n+1)-1} \nonumber \\
                &=& (-1)^{n+1}(n+1)!x^{-((n+1)+1)} \nonumber 
  \end{eqnarray}
  Ce qui prouve le résultat.
\end{itemize}

\textit{Remarque:} il est utile de calculer les premières dérivées à la main pour avoir l'intuition du résultat.
