a) En prenant $x=y=0$, il vient $f(0)=2f(x)$, donc $f(0)=0$. Puis, pour tout $x\in\mathbb{R}$, on a $0=f(0)=f(x+(-x))=f(x)+f(-x)$, donc $f(-x)=-f(x)$, ce qui prouve que $f$ est impaire.\\ \\
b) Montrons par récurrence que pour tout $n\in\mathbb{N}$, et pour tout $x\in\mathbb{R}$, $f(nx)=nf(x)$:
\begin{itemize}
  \item \textit{Initialisation}: Fixons $n=0$. Quel que soit $x\in\mathbb{R}$, on a bien $f(0\cdot x)=0\cdot f(x)$
  \item \textit{Hérédité}: Supposons la propriété vraie au rang $n\in\mathbb{N}$. Alors, pour tout $x$ réel:
  \[
  f((n+1)x) = f(nx)+f(x) = nf(x)+f(x) = (n+1)f(x)
  \]
  ce qui achève la preuve.
\end{itemize}
Puisque $f$ est impaire, le résultat peut être étendu sur $\mathbb{Z}$: 
\[ \forall k\in\mathbb{Z}, \forall x \in \mathbb{R}, \quad f(kx)=kf(x) \]
en effet, si $k<0$, alors pour tout $x$ réel $f(kx)=-f((-k)x)=kf(x)$.\\ \\
c) Soit $q\in\mathbb{N}^*$, d’après le point b) on remarque que: 
  \[ f\left( 1 \right) = f\left( q \times \frac{1}{q} \right) = qf\left(\frac{1}{q}\right) \]
ce qui implique $f\left(\frac{1}{q}\right) = \frac{f(1)}{q}$. Soit $x\in\mathbb{Q}$: il existe deux entiers $p\in\mathbb{Z}$ et $q\in\mathbb{N}^*$ tels que $x=\frac{p}{q}$, et alors:
\[
f\left(\frac{p}{q}\right) = pf\left(\frac{1}{q}\right) = \frac{p}{q}f(1) = ax
\]
d) Tout réel s’écrit comme limite d'une suite de nombres rationnels car $\mathbb{Q}$ est dense dans $\mathbb{R}$: en particulier, tout intervalle (non-vide) de $\mathbb{R}$ contient au moins un rationnel\footnote{et même une infinité, mais un seul nous suffit ici!}. Soit $x\in\mathbb{R}$: alors pour tout $n\in\mathbb{N}^*$, l'intervalle $[x-1/n, x+1/n]$ contient un rationnel $r_n$, et $|r_n-x|\leq \frac{1}{n}$ donc $r_n\to x$.\\ \\
e) Soit $x\in\mathbb{R}$. D’après d), il existe une suite de rationnels $(x_n)$ qui tend vers $x$. D’après c), pour tout $n\in\mathbb{N}$, $f(x_n)=ax_n\to ax$, et comme $f$ est continue, $f(x_n)\to f(x)$. Par unicité de la limite, on conclut que $f(x)=ax$.
