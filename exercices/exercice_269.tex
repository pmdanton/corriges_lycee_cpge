\`A chaque réalisation $(x,y,z)$ dont la somme vaut $s\leq 10$, on peut associer la réalisation "opposée" $(7-x,7-y,7-z)$ dont la somme vaut $21-s$, donc 11 ou plus: on en déduit qu’il est autant probable d'obtenir 10 ou moins que 11 ou plus. Comme tous les résultats possibles sont couverts, on en déduit que cette probabilité commune vaut $\frac{1}{2}$.\\ \\
\textit{Remarque}: cette utilisation de la symétrie rappelle un peu l'astuce du jeune Gauss, qui calcula $S=1+2+\cdots+99+100$ en associant chaque $n$ \`a $101-n$, pour obtenir $2S=101+101+\cdots+101+101=10100$ donc $S=5050$. D'une manière générale, les arguments de symétrie permettent souvent d’éviter les calculs en probabilités.\\ \\
\textit{Pour aller plus loin}: Généraliser \`a $n\in\mathbb{N}^*$ dés lancés ensemble.
