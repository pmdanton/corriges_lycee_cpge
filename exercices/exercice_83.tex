Pour $m\in\mathbb{R}$, le discriminant vaut:
\[
\Delta(m)=m^2-4=(m-2)(m+2)
\]
Donc:
\begin{itemize}
  \item Si $m\in\{-2,+2\}$, $\Delta(m)=0$ et $p_m$ admet une racine réelle double.
  \item Si $|m|<2$, $\Delta(m)<0$ et $p_m$ n'admet aucune racine réelle.
  \item Si $|m|>2$, $\Delta(m)>0$ et $p_m$ admet deux racines réelles distinctes.
\end{itemize}
