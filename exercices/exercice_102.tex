a) Calculons:
\begin{eqnarray*}
  \cos(2\alpha) &=& 2\cos^2(\alpha)-1 \\
  &=& 2\left( \frac{\sqrt{5}-1}{4} \right)^2 -1 \\
  &=& \frac{3-\sqrt{5}}{4} -1 \\
  &=& -\frac{1+\sqrt{5}}{4} 
\end{eqnarray*}
Puis:
\begin{eqnarray*}
  \cos(4\alpha) &=& 2\cos^2(2\alpha)-1 \\
  &=& 2\left( \frac{1+\sqrt{5}}{4} \right)^2 -1 \\
  &=& \frac{3+\sqrt{5}}{4} -1 \\
  &=& \frac{\sqrt{5}-1}{4} 
\end{eqnarray*}
b) On a $\cos(\alpha)=\cos(4\alpha)$, ce qui implique bien $4\alpha = \pm \alpha$ modulo $2\pi$. \\
c) D’après b), il existe $k\in \mathbb{Z}$ tel que $4\alpha = \pm \alpha + 2k\pi$, donc $\alpha \in \left\lbrace \frac{2k\pi}{5}, \frac{2k\pi}{3} \right\rbrace$.  Comme on cherche $\alpha \in [0,\pi]$, on a $\alpha \in \left\lbrace 0, \frac{2\pi}{5}, \frac{2\pi}{3}, \frac{4\pi}{5} \right\rbrace$. Comme on sait que $\frac{\pi}{2} \leq  \frac{2\pi}{3} \leq \frac{4\pi}{5} \leq \pi $, on peut exclure $ \frac{2\pi}{3}, \frac{4\pi}{5} $ dont le cosinus est négatif. Enfin, comme $\cos(0)=1$, il ne reste que $\alpha = \frac{2\pi}{5}$.
